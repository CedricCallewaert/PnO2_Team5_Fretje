\documentclass[kulak]{kulakarticle} % options: kulak (default) or kul
\usepackage[dutch]{babel}
\title{Financieel rapport}
\author{ \\Cedric Callewaert, Richard Nollet 
	Carlo Matvij,\\ Maxime Van Insberghe, Tim Wyckaert}
\usepackage{pdfpages}

\begin{document}

\maketitle
	
	\section{Motivering materiaalkeuze}
	Om onze brandblusser optimaal te ontwerpen, is er uiteraard het juiste materiaal nodig. Met een budjet van 3500 virtuele credits was er keuze uit een lijst van beschikbare materialen en hardware. Door de beperking van het budjet moesten we dus ook wat voor- en nadelen afwegen om op basis hiervan een selectie uit de lijst aan te kopen.	Er waren echter ook maar een beperkt aantal stukken van elk materiaal beschikbaar, waardoor de aankoop niet te lang mocht worden uitgesteld.
	Bepaalde aankopen zijn voor de hand liggend, andere worden hieronder gemotiveerd.
	
	\begin{itemize}
	\item De servo motors hebben we aangekocht om een verticale en horizontale beweging mogelijk te maken van onze waterstraal.
	
	\item We hebben de Myrio verkozen boven de Raspberry Pi omdat deze eenvoudiger zou zijn om mee te werken.
	
	\item Het LED-lampje dient om te weten of het brandblusapparaat actief is.
	
	\item de USB Webcam 1080P hebben werd verkozen boven de USB Webcam Type 2 omdat deze een groter bereik heeft. 
	
	\item Voor de pomp is de krachtigste de beste keuze om een groter bereik te halen met onze waterstraal. 
	
	\item De step-Down Voltage Regulator dient ervoor om hardware die op verschillende voltages werken in eenzelfde elektrische kring te kunnen schakelen.
	
	\item Het MDF-hout dient ervoor om een bak te maken waarin de elektronica kan geplaatst worden en ook om de brandblusser zelf op te monteren.
	
	\item Al de 3D-geprinte bestellingen zijn materialen die specifiek voor het ontwerp van de brandblusser zijn en dus niet in de lijst staan waaruit we kunnen kiezen. Dit zijn tandwielen, mondstukken voor de waterstraal te optimaliseren en verstevigingen. 
	\end{itemize}
	breadboard, voeding, magneetventiel, relais, 
	
	\section{Overzicht financiële uitgaven}
	
\includepdf[page={1-2}]{financiële_tabel}
	
	
	
	
\end{document}