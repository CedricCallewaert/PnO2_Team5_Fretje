\documentclass{kulakarticle}
\usepackage{graphicx} % Required for inserting images
\usepackage[dutch]{babel}
\title{Teamvergadering 8}
\author{C. Callewaert, R. Nollet \\
	C. Matvij, M. Van Insberghe, T. Wyckaert }
\date{Academiejaar 2022 -- 2023}
\address{
	\textbf{Groep Wetenschap \& Technologie Kulak} \\
	Naam van de opleiding \\
	Naam van het vak}
\begin{document}
	\maketitle
	\section{Verslag begin sessie}
	
	
	
	\subsection{Evaluatie activiteiten}
	
 Deze sessie waren we van plan om te beginnen solderen met de printstukken die we verstuurt hadden. We hebben de bak waarin we onze uiteindelijke brandblusser gaan op monteren thuis al grotendeels in elkaar gezet. De printstukken zijn echter nog niet klaar, waardoor we onze planning zullen moeten aanpassen. normaal zouden alle stukken in solid edge nu moeten getekend zijn. De coördinatenomzetting is nog altijd niet opgelost.
	
	\subsection{Planning}
	
	Omdat we verlopig wat minder werk voor handen hebben, zullen we ons met wat meer focussen om de coördinatenomzetting op te lossen. Voor de rest kunnen we ons bezighouden met al wat te beginnen aan het eindverslag en de powerpoint. Ook zullen we vandaag allemaal ons deel aan elkaar uitleggen zodat we helemaal op de hoogte zijn van elkaars deel. Ook kan het financieel rapport afgewerkt worden aangezien we nu al ons materiaal hebben en hoogswaarschijnlijk niets meer zullen bestellen.
	
	\section{Verslag einde sessie}
	

	
	
	\subsection{Bespreking verslag vorige vergadering}
	
		Op het einde van deze sessie hebben we ons probleem met de coördinaten kunnen achterhalen en nu weten we hoe we het moeten oplossen. We zijn echter wel op een serieus probleem gestoten. Onze servomotor kan hoogstwaarschijnlijk niet genoeg kracht geven om ons platform te laten draaien. Aangezien we geen servomotors meer kunnen komen, zullen we hiervoor een origineel alternatief moeten zoeken.
	
	
	\subsection{Planning}
	We zullen al een stuk van het eindverslag proberen af te werken tegen de volgende sessie. Iedereen heeft verder ook een taak om aan te werken en heeft zijn handen vol. Ook zullen we proberen samen te komen op een bepaald moment in de week om de voortgang te bespreken en eventueel nog aansturingen te doen.
	
	
	
\end{document}