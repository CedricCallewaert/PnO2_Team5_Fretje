\documentclass{kulakarticle}
\usepackage{graphicx} % Required for inserting images
\usepackage[dutch]{babel}
\title{Teamvergadering 5}
\author{C. Callewaert, R. Nollet \\
	C. Matvij, M. Van Insberghe, T. Wyckaert }
\date{Academiejaar 2022 -- 2023}
\address{
	\textbf{Groep Wetenschap \& Technologie Kulak} \\
	Naam van de opleiding \\
	Naam van het vak}
\begin{document}
	\maketitle
	\section{Verslag begin sessie}
	Het mondstuk voor op de buis is getekend, de mathematische berekeningen voor de hoeken van de waterstraal te berekenen blijven een lastig probleem. Voor de rest zijn er nog niet veel aanpassingen gebeurt.

	\subsection{Planning}
	Vandaag zijn we van plan om weer de myRio verder te initialiseren. We zullen hiermee de motor testen. Ook de pomp zullen we nog eens testen aangezien dit de vorige keer niet zo goed gelukt was. De berekeningen rond de hoeken voor de waterstraal zullen nog verder uitgewerkt moeten worden. Ook zullen we ons bezig houden met de cameracode van Python naar Labview over te zetten. Voor het mondstuk aan onze brandblusser te monteren moeten er wat aanpassingen gemaakt worden zodat het er goed op past. We zullen ook beginnen aan ons eindverslag en aan de powerpoint die we volgende week moeten indienen en presenteren.
	
	
	\section{Verslag einde sessie}
	

	
	\subsection{Bespreking verslag vorige vergadering}
	
		Ons werk is goed gevorderd. Wegens de test van een ander team zijn we te weten gekomen dat ons mondstuk nog aangepast moet worden om verder te kunnen spuiten en een meer laminaire stroom te creëren zodat we een precisere straal hebben. Deze hebben we dan ook kunnen afwerken binnen de sessie.We zijn nu ook al begonnen aan het tekenen van de tandwielen die we willen gebruiken. De camera is nu geprogrammeerd in Labwiew. De werking van de motors door aansturing van de computer en de myRio is ook al goed gevorderd. Bij de mathematische berekeningen zitten we nog altijd redelijk vast. Normaal kunnen we deze in python implementeren zodat de computer de rest van de calculatie uitrekent. De inleiding en de planning van ons project staan al in het eindverslag. enkel het ontwerpproces moet hier nog in verwerkt worden.
	
	
	\subsection{Planning}
	
	Een teamlid zal thuis proberen verder te werken aan de myRio. Verder zal er ook nog iemand verder werken aan de tandwielen die we willen 3D-printen. Ook zullen verder werken aan de powerpoint en het verlopige verslag, want deze moeten volgende week ingediend worden. 

	
\end{document}