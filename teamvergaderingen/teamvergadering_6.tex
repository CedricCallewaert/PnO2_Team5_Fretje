\documentclass{kulakarticle}
\usepackage{graphicx} % Required for inserting images
\usepackage[dutch]{babel}
\title{Teamvergadering 6}
\author{C. Callewaert, R. Nollet \\
	C. Matvij, M. Van Insberghe, T. Wyckaert }
\date{Academiejaar 2022 -- 2023}
\address{
	\textbf{Groep Wetenschap \& Technologie Kulak} \\
	Naam van de opleiding \\
	Naam van het vak}
\begin{document}
	\maketitle
	\section{Verslag begin sessie}
	
	\subsection{Evaluatie activiteiten}
	We zijn erin geslaagd om de mathematische berekeningen van de waterstraal te vinden. Zo kunnen we bepalen welke beginsnelheid het water moet hebben om de maximale afstand te halen. Hierdoor kunnen we ook de straal van ons mondstuk berekenen. Onze presentatie is ook afgewerkt en klaar om voorgesteld te worden. Ons Eindverslag is qua inhoud af, maar er zal nog wat moeten gewerkt worden aan de bronvermelding en lay-out.
	 
	
	\subsection{Planning}
	Op onze planning voor vandaag staat uiteraard de presentatie voorbrengen en onze tussentijdse verslag afwerken en indienen. Verder moet het cameraprobleem aangepakt worden dat we recent zijn tegengekomen. De camera heeft namelijk een te klein verticaal bereik. Ook wordt het nog een uitdaging om de juiste coördinaten door te geven aan onze code die de motor moet aansturen op basis van de x- en y-waarden. Verder gaat er nog een beetje geassembleerd worden aan ons ontwerp.

	
	\section{Verslag einde sessie}
	
	\subsection{Bespreking verslag vorige vergadering}
	Na onze presentatie hebben we enkele tips meegekregen om ons ontwerp aan te passen en onze volgende presentatie te verbeteren. Hier zullen we uiteraard rekening mee houden. Het probleem dat de camera een te klein gezichtsveld heeft is opgelost doordat we onze camera mochten inruilen met een camera met een breder gezichtsveld. Het tussentijdse verslag is ingediend.
	\subsection{Planning}
%	Met de paasvakantie die komt is het belangrijk om het werk goed te verdelen. De assemblage zal moeten wachten tot na de vakantie aangezien dit het eenvoudigst is om op school te doen. Verder moet er vooral nog code geschreven worden om de camera goed de initialiseren. Ook moet er nog tijd gespendeerd worden het probleem van de coördinaatbepaling. Dit lijkt veel tijd in te nemen, vooral in opzoekwerk. Er zullen ook verschillende mondstukken ontworpen worden zodat we de beste kunnen uitkiezen na het testen.


	

	
\end{document}