\documentclass{kulakarticle}
\usepackage{graphicx} % Required for inserting images
\usepackage[dutch]{babel}
\title{Eindverslag: De slimme brandblusser}
\author{C. Callewaert, R. Nollet \\
	 C. Matvij, M. Van Insberghe, T. Wyckaert }
\date{Academiejaar 2022 -- 2023}
\address{
	\textbf{Groep Wetenschap \& Technologie Kulak} \\
	Naam van de opleiding \\
	Naam van het vak}
\begin{document}
	
\maketitle

\tableofcontents
	
\section*{Inleiding}

In het bedrijfsleven zijn er enorm veel wetten en regels om de veiligheid te optimaliseren. Brandveiligheid is hiervan een van de belangrijkste. Het aanwezig zijn van een automatisch brandblussysteem in een bedrijf is dan ook verplicht. Het populairste is het sprinklersysteem. Deze heeft een groot bereik en kan grote gebouwen dus eenvoudig en efficiënt blussen. Het nadeel aan dit systeem is echter dat er veel installatiekosten aan te pas komen. Ook zijn er veel onderhoudswerken nodig om het sprinklersysteem functioneel te houden. Daarom wordt een goedkoper alternatief gezocht zonder daarbij de brandveiligheid te verminderen. Het ontwerp van de "slimme" brandblusser komt hiervoor aan de pas. Deze brandblusser detecteert brand met behulp van een camera en kan automatisch een waterstraal richten op de brand en deze beginnen blussen. Om  de brandblusser op de markt te brengen, moet deze wel nog uitgebreid getest worden omdat het idee nog in zijn kinderschoenen staat. In dit artikel bespreken we het ontwerp, de bouw en de testresultaten van de slimme brandblusser. 

\section{Planning}

Omdat we bij dit project een deadline hebben gekregen, is het belangrijk om een goede planning te maken zodat we tijdig klaar geraken. Hiervoor hebben we gebruik gemaakt van Gantt Chart. Dit is een \LaTeX-functie die toelaat om de deeltaken zorgvuldig weer te geven met de bijhorende deadline. Zo kunnen we een goed overzicht bewaren over de actieve taken, hoelang er nog aan gewerkt zal worden en zo worden er geen taken dubbel gemaakt. Ook zijn we zo op de hoogte van de voortgang van het project.  De taken van de Gantt Chart worden verdeeld onder de teamleden. Als een taak volbracht is, is het eenvoudig  te zien aan welke taak je vervolgens kan beginnen werken.

De belangrijkste taken die te tijd zullen vullen zijn de mathematische berekeningen, het programmeren van de brandblusser en het tekenen van ons ontwerp. Het zijn deze taken die onderverdeeld worden in deeltaken en in de Gantt Chart gepland zijn. Eerst hebben de klantenvereisten overlopen om te weten waaraan de brandblusser moet voldoen. Omdat we niet kunnen beginnen ontwerpen zonder te weten welk materiaal er te beschikking staat, is het logisch om hierna eerst te kijken wat er in de aanbieding staat. Om een idee te hebben over het ontwerp, is een eerste schets van de brandblusser nodig. Dit is al een behoorlijk accuraat ontwerp waaruit we kunnen afleiden welke materialen en software we zullen nodig hebben om ons idee te realiseren.  Hierna zal onder andere het programmeren en het tekenen van de nodige onderdelen die we moeten 3D-printen beginnen.
Omdat er waarschijnlijk nog wijzigingen zullen zijn, zal het printen zelf nog even op zich laten wachten zodat we geen credits en materiaal verspillen. De mathematische berekeningen rond de waterstraal zullen ook tijd vergen, net zoals de hoeken die onze waterstraal moet maken om de brand te blussen.  Ook moet onze camera geïnitialiseerd worden zodat deze rood herkent.  Omdat deze taken chaotisch door elkaar kunnen lopen,  zal ieder teamlid zich op een van deze taken focussen. 





\section{Ontwerpproces}



\subsection{Solid Edge}

\subsection{MyRio}


\subsection{Camera initialiseren}

\subsection{Mathematische berekeningen}

Onder mathematische berekeningen worden vooral de calculaties omtrent de waterstraal die uit de brandblusser komt bedoelt. Zo moet bijvoorbeeld op basis van het mondstuk die op de waterbuis gemonteerd staat en de druk, het bereik van de waterstraal moeten kunnen berekend worden. Ook moeten de hoeken bepaald w
 

\section*{bronnen}
https://website.nbn.be/themas/brandveiligheid




\end{document}





   