\documentclass{kulakarticle}
\usepackage{graphicx} % Required for inserting images
\usepackage[dutch]{babel}
\title{Eindverslag: De slimme brandblusser}
\author{C. Callewaert, R. Nollet \\
	 C. Matvij, M. Van Insberghe, T. Wyckaert }
\date{Academiejaar 2022 -- 2023}
\address{
	\textbf{Groep Wetenschap \& Technologie Kulak} \\
	Naam van de opleiding \\
	Naam van het vak}
\begin{document}
	
\maketitle

\tableofcontents
	
\section*{Inleiding}

In het bedrijfsleven zijn er enorm veel wetten en regels om de veiligheid te optimaliseren. Brandveiligheid is hiervan een van de belangrijkste. Het aanwezig zijn van een automatisch brandblussysteem in een bedrijf is dan ook verplicht. Het populairste is het sprinklersysteem. Deze heeft een groot bereik en kan grote gebouwen dus eenvoudig en efficiënt blussen. Het nadeel aan dit systeem is echter dat er veel installatiekosten aan te pas komen. Ook zijn er veel onderhoudswerken nodig om het sprinklersysteem functioneel te houden. Daarom wordt een goedkoper alternatief gezocht zonder daarbij de brandveiligheid te verminderen. Het ontwerp van de "slimme" brandblusser komt hiervoor aan de pas. Deze brandblusser detecteert brand met behulp van een camera en kan automatisch een waterstraal richten op de brand en deze beginnen blussen. Om  de brandblusser op de markt te brengen, moet deze wel nog uitgebreid getest worden omdat het idee nog in zijn kinderschoenen staat. In dit artikel bespreken we het ontwerp, de bouw en de testresultaten van de slimme brandblusser. 

\section{Planning}

Omdat we bij dit project een deadline hebben gekregen, is het belangrijk om een goeie planning te maken zodat we tijdig klaar geraken. Hiervoor hebben we gebruik gemaakt van Gantt Chart. Dit is een \LaTeX-package die toelaat om de deeltaken zorgvuldig weer te geven met de bijhorende deadline. Zo kan je een goed overzicht bewaren over de actieve taken en hoelang er nog aan gewerkt zal worden. Ook ben je zo op de hoogte van de voortgang van het project.

\subsection{Takenverdeling}




\subsection{Voortgang}



\section{Ontwerpproces}

\subsection{Solid Edge}

\subsection{MyRio}


\subsection{Camera initialiseren}

\subsection{Mathematische berekeningen}


 

\section*{bronnen}
https://website.nbn.be/themas/brandveiligheid

\end{document}





   