\documentclass{kulakarticle}
\usepackage{graphicx} % Required for inserting images
\usepackage[dutch]{babel}
\title{Eindverslag: De slimme brandblusser}
\author{C. Callewaert, R. Nollet \\
	 C. Matvij, M. Van Insberghe, T. Wyckaert }
\date{Academiejaar 2022 -- 2023}
\address{
	\textbf{Groep Wetenschap \& Technologie Kulak} \\
	Naam van de opleiding \\
	Naam van het vak}
\begin{document}
	
\maketitle
	
\section*{Inleiding}

In het bedrijfsleven zijn er enorm veel wetten en regels om de veiligheid te optimaliseren. Brandveiligheid is hiervan een van de belangrijkste. Het aanwezig zijn van een automatisch brandblussysteem in een bedrijf is dan ook verplicht. Het populairste is het sprinklersysteem. Deze heeft een groot en bereik kan grote gebouwen dus eenvoudig en efficiënt blussen. Het nadeel aan dit systeem is echter dat er veel installatiekosten aan te pas komen. Ook zijn er veel onderhoudswerken nodig om het sprinklersysteem functioneel te houden. Daarom wordt een goedkoper alternatief gezocht zonder daarbij de brandveiligheid te verminderen. Het ontwerp "slimme" brandblusser komt hiervoor aan de pas. Deze brandblusser detecteert brand met behulp van een camera en zal automatisch een waterstraal richten op de brand en beginnen blussen. Om  de brandblusser op de markt te brengen, moet deze wel nog uitgebreid getest worden. In dit artikel bespreken we het ontwerp, de bouw en de testresultaten van de slimme brandblusser. 


\section*{bronnen}
https://website.nbn.be/themas/brandveiligheid

\end{document}





   