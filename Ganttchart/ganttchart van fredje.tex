\documentclass[landscape,12pt]{article}

% style
\usepackage[left=2.5cm,top=2cm,right=2.5cm,bottom=2cm,b2paper]{geometry}
\usepackage{fancyhdr}
\pagestyle{fancy}
\lhead{Team Fredje}
\rhead{Gantt chart}
\renewcommand{\headrulewidth}{0.4pt}
\usepackage{color}

% gantt
\usepackage{pgfgantt}
\def\pgfcalendarweekdayletter#1{%
	\ifcase#1M\or T\or W\or T\or F\or S\or S\fi%
}

\definecolor{foobarblue}{RGB}{0,153,255}
\definecolor{foobaryellow}{RGB}{234,187,0}
\definecolor{grey}{RGB}{170,170,170}

\newganttchartelement{foobar}{
	foobar/.style={
		shape= rectangle,
		inner sep=0pt,
		draw=grey!70!blue,
		thick,
		fill=white,
		inner color=blue!10, outer color=blue!40, opacity=0.95
	},
	foobar incomplete/.style={
		/pgfgantt/foobar,
		draw=foobaryellow,
		bottom color=foobaryellow!50
	},
	foobar label font=\slshape,
	foobar left shift=0,%.1,
	foobar right shift=0%-.1
}

\begin{document}
	
	% Een goede Gantt-chart maakt gebruik van \emph{links} en \emph{milestones}. Deze laatste definieer je in een lijstje onder de Gantt chart.
	
	\begin{figure}
		\centering
		\begin{ganttchart}[hgrid, vgrid, x unit=5mm, time slot format=isodate]{2023-02-17}{2023-05-26}
			\gantttitlecalendar{week,month=shortname,day,weekday=letter} \\
			
			%\ganttgroup{2}{2023-02-17}{2023-03-31} \\
			\ganttfoobar[name= 1.0]{1.0- Documenten lezen}{2023-02-17}{2023-02-17} \\
			\ganttfoobar[name=1.1]{1.1- Model brainstorm}{2023-02-17}{2023-03-03} \\
			%\ganttlink{11}{12} % 21 must be completed before 22
			\ganttfoobar[name=1.2]{1.2- Schets model}{2023-02-17}{2023-02-24} \\
			%\ganttlink{12}{13} 
			\ganttfoobar[name=1.3]{1.3 -Documenteren}{2023-02-27}{2023-03-31} \\
			\ganttfoobar[name=2.0]{2.0- Modelleren}{2023-03-03}{2023-04-21} \\
			\ganttfoobar[name=2.1]{2.1 - Materialen kiezen}{2023-03-3}{2023-04-21} \\
			%\ganttlink{2.1}{12} % 21 must be completed before 22
			\ganttfoobar[name=2.2.1]{2.2.1 - Chassis}{2023-03-17}{2023-03-31} \\
			\ganttfoobar[name=2.2.2]{2.2.2 - Gimbal}{2023-03-24}{2023-03-31} \\
			\ganttfoobar[name=2.2.3]{2.2.3 - Omhulsel elektronica}{2023-04-28}{2023-05-05} \\
			\ganttfoobar[name=2.3]{2.3 - Assembly in SE}{2023-03-17}{2023-05-05} \\
			\ganttfoobar[name=3.0]{3.0 - Elektronica}{2023-03-17}{2023-05-12} \\
			\ganttfoobar[name=3.1]{3.1 - Elektrisch netwerk}{2023-03-17}{2023-05-26} \\
			\ganttfoobar[name=3.2]{3.2 - Experimenteren}{2023-04-14}{2023-05-26} \\
			\ganttfoobar[name=3.3]{3.3 - Aanbrengen elektronica}{2023-04-14}{2023-04-28} \\
			\ganttfoobar[name=4.0]{4.0 - Programmeren}{2023-03-31}{2023-04-28} \\
			\ganttfoobar[name=4.1]{4.1 - Stuurprogramma's motoren}{2023-03-24}{2023-04-21} \\
			\ganttfoobar[name=4.2]{4.2 - Camera detectie LabVIEW}{2023-03-10}{2023-05-05} \\
			\ganttfoobar[name=4.3]{4.3 - Grafische gebruikers interface}{2023-03-24}{2023-03-31} \\
			\ganttfoobar[name=5.0]{5.0 - Berekeningen}{2023-03-03}{2023-03-31} \\
			\ganttfoobar[name=5.1]{5.1 - Waterstraalbereik}{2023-03-03}{2023-03-24} \\
			\ganttfoobar[name=5.2]{5.2 - Hoek}{2023-03-10}{2023-03-31} \\
			\ganttfoobar[name=6.0]{6.0 - Schrijven artikel}{2023-03-10}{2023-05-26} \\
			\ganttfoobar[name=7.0]{7.0 - Assembleren}{2023-04-28}{2023-05-12} \\
			\ganttmilestone{Bieding materialen}{2023-03-10} \\ % milestone completed
			\ganttmilestone{Presentatie 1 voorbereiden}{2023-03-31} \\ % milestone completed
			\ganttmilestone{Individuele vraag}{2023-05-12} \\ % milestone completed
			\ganttmilestone{Demo}{2023-05-19} \\ % milestone completed
			\ganttmilestone{Presentatie 2 voorbereiden}{2023-05-26} \\% milestone completed
			
		\end{ganttchart}
	\end{figure}
	
\end{document}
