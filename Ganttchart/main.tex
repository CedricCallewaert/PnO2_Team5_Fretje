\documentclass[landscape,12pt]{article}
  
% style
\usepackage[left=2.5cm,top=2cm,right=2.5cm,bottom=2cm,a4paper]{geometry}
\usepackage{fancyhdr}
\pagestyle{fancy}
\lhead{Team Fredje}
\rhead{Gantt chart}
\renewcommand{\headrulewidth}{0.4pt}
\usepackage{color}

% gantt
\usepackage{pgfgantt}
\def\pgfcalendarweekdayletter#1{%
	\ifcase#1M\or T\or W\or T\or F\or S\or S\fi%
}

\definecolor{foobarblue}{RGB}{0,153,255}
\definecolor{foobaryellow}{RGB}{234,187,0}
\definecolor{grey}{RGB}{170,170,170}

\newganttchartelement{foobar}{
	foobar/.style={
		shape= rectangle,
		inner sep=0pt,
		draw=grey!70!blue,
		thick,
		fill=white,
		inner color=blue!10, outer color=blue!40, opacity=0.95
	},
	foobar incomplete/.style={
		/pgfgantt/foobar,
		draw=foobaryellow,
		bottom color=foobaryellow!50
	},
	foobar label font=\slshape,
	foobar left shift=0,%.1,
	foobar right shift=0%-.1
}

\begin{document}

% Een goede Gantt-chart maakt gebruik van \emph{links} en \emph{milestones}. Deze laatste definieer je in een lijstje onder de Gantt chart.

	\begin{figure}
		\centering
		\begin{ganttchart}[hgrid, vgrid, x unit=5mm, time slot format=isodate]{2023-02-17}{2023-03-31}
			\gantttitlecalendar{week,month=shortname,day,weekday=letter} \\
			
			%\ganttgroup{2}{2023-02-17}{2023-03-31} \\
			\ganttfoobar[name=11]{1.1/1.2}{2023-02-17}{2023-02-17} \\
			\ganttfoobar[name=12]{1.2}{2023-02-17}{2023-02-17} \\
			%\ganttlink{11}{12} % 21 must be completed before 22
			\ganttfoobar[name=13]{1.3}{2023-02-17}{2023-02-17} \\
			%\ganttlink{12}{13} 
			\ganttfoobar[name=20]{2.0}{2023-02-27}{2023-03-31} \\
            \ganttfoobar[name=21]{3.1}{2023-03-17}{2023-03-31} \\
            \ganttfoobar[name=22]{4.0}{2023-03-10}{2023-03-31} \\
            \ganttmilestone{1.4 - Lijst onderdelen}{2023-03-3} \\% milestoncompleted
            \ganttmilestone{6.1 - Presentatie 1}{2023-03-31} % milestone completed
            %\ganttmilestone{6.2 - Presentatie 2}{2023-05-26} % milestone completed
			
		\end{ganttchart}
	\end{figure}

\end{document}
