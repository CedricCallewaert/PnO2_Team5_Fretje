\documentclass{kulakarticle}
\usepackage{graphicx} % Required for inserting images
\usepackage[dutch]{babel}
\title{klantenvereisten en ontwerpspecificaties}
\author{C. Callewaert, R. Nollet, C. Matvij, M. Van Insberghe, T. Wyckaert }
\date{March 2023}

\begin{document}

\maketitle

\section{Klantenvereisten}

Ons doel is om een brandblusser te ontwerpen die een alternatief moet vormen voor sprinklersystemen die grote gebouwen tot op dit moment brandveilig maken. Deze zijn namelijk duur, hebben veel onderhoudswerken en zijn moeilijk te installeren. Ons ontwerp is een slimme brandblusser die brand kan blussen zonder manuele besturing. Deze brandblusser kan aan het plafond geïnstalleerd worden en kan via een camera  de brand  detecteren. Het grote voordeel aan dit systeem  is dat er geen leidingen moeten worden aangelegd over het hele gebouw, een enkele volstaat. Dit zal de installatiekosten in grote mate laten afenemen. Ook zijn er zo minder onderhoudswerken omdat er maar  enkele leidingen en brandblusapparaten moeten gecontroleerd worden. Dit maakt deze brandblusser bijgevolg een stuk goedkoper. Ook zal de efficiëntie hoger zijn. De slimmer brandblusser richt zijn straal op de brand en zal dus niet de hele verdieping van het gebouwen nat maken, wat handig kan zijn indien er  bijvoorbeeld elektronica in het gebouw aanwezig is. Er zal wel elektronica in de brandblusser aanwezig zijn, waardoor er gezorgd moet worden dat deze goed afgeschermd is van het water. Voor de veiligheid moet er natuurlijk een noodstop aanwezig zijn.


\section{Ontwerpspecificaties}

De slimme brandblusser zal gemonteerd worden aan een plaat van 0.75 op 0.75 meter. Deze zal dan vervolgens aan het plafond opgehangen worden. Voor de proeftest moet de brandblusser een gebied van zes meter breed en zeven meter lang op minstens drie meter afstand kunnen bestrijken. De waterstraal moet dus een minimale horizontale  hoek van 90 graden kunnen maken. Deze rotatie zal door een servo motor worden aangestuurd. De motor laat een klein tandwiel draaien, die dan opnieuw geconnecteerd is met een groter tandwiel die vaststaat op een rotatieas. Zo zal de snelheid van de rotatie verkleinen, maar zal de kracht groter worden waardoor de motor een zwaardere massa kan verplaatsten.  
Om tien meter  ver te spuiten kan, afhankelijk van de sterkte van de pomp, de straal van de buis waaruit het water komt kleiner of groter gemaakt worden. Omdat we met constante druk werken, zal de buis ook een verticale beweging moeten kunnen maken. Dit doen we met een tweede servo motor die twee tandwielen laat draaien door deze met een bandketting te verbinden. 
 
Ons brandblusapparaat hangt aan het plafond, dus het is ook belangrijk om de massa zo klein mogelijk te houden. Voor de test zullen we ook niet met leidingen werken, maar met een bidon van 10 liter wegens praktische redenen. Dit bemoeilijkt het natuurlijk om de massa te reduceren, maar dit extra gewicht zou geen problemen mogen veroorzaken. 
Om stroom te leveren zullen we een standaard Belgisch stopcontact van 230 volt.
Om de elektronica af te schermen van het water, gaan we de waterpomp en de bidon naast de motoren positioneren en de elektronica vervolgens afschermen met een halve koepel die zal bestaan uit plastic of een ander materiaal. 

De brandblusser zal bestuurd worden via een computer die met het toestel verbonden is via een kabel.

\end{document}
